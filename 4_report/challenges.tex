%%%%%%%%%%%%%%%%%%%%%%%%%%%%%%%%%%%%%%%%%%%%%%%%%%%%%%%%%%%%%%%%%%%%%%%%%%%%%%%%
%
%	CHALLENGES FACED
%
%%%%%%%%%%%%%%%%%%%%%%%%%%%%%%%%%%%%%%%%%%%%%%%%%%%%%%%%%%%%%%%%%%%%%%%%%%%%%%%%
\label{sec:challenges}

\begin{itemize}
Throughout the course of this project, we faced several difficulties which 
shaped the course of our work:
\item The greatest challenge faced by us was to ensure the intended mapping
of branches into intrinsic and extrinsic. Though we had labelled much more 
benchmarks, the number of static branches didn't equal the number of labels for most of them.
Most of the time we encountered compounded boolean statements which did not assemble to branches
predictably. We believe the compiler's decisions on cost-analysis of merging two or more branches
was the most difficult thing to predict, and we were unable to get many benchmarks which could
agree with all our assumptions. Notably, none of the benchmarks that we tagged in \cite{parboil-benchmark-suite}
agreed with our assumptions. Therefore, we were compelled to chose our Benchmarks from
an entirely different suite than the one we had planned.
\item We lost a considerable time struggling to understand the code written in benchmark applications. 
We believe a valuable amount of time was lost simply trying
to understand the meaning of variables and functions, which could be easily avoided 
by profuse comments and explanations.
\item Our initial plan was to instrument the CPU code of the relevant benchmarks, identify
their branches as intrinsic and extrinsic, and compare their behaviour with those observed
in our GPU simulations. Our aim was to investigate whether intrinsic HLC structures were similar
in both GPU and CPU. However, we couldn't figure out a way to preserve our branch identification tokens
across the compilation processes. Branches from library functions or inlining of conditionals
would make even the simplest HLC structures unpredictable.

\end{itemize}

