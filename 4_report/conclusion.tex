
%----------------------------------------------------------------------------------------
%	CONCLUSION and FUTURE SCOPE
%----------------------------------------------------------------------------------------

\label{sec:conclusion}
\par{
\textbf {Extrinsic branches seem to cause significant Inactivity.} The results from Section \ref{sec:results}
, though not exhaustive, suggest that extrinsic branches may cause large inactivity in some benchmarks. However, it isn't clear
that the inactivity is a result of the nature of such branches, or rather of inefficient programming
practises. 

While learning the algorithm in order to understand and label the HLC structures, we often
came across extra branches in critical sections of the code (for example, in a very important function, 
in the main iterative loop, or the core of the kernel algorithm) present just to support an extra queue
or array that made the program easier to write. We believe such unnecessary branches hurt performance
significantly, and can be easily removed (albeit making the code less readable and maintainable). 

Hence, a natural extension of this project could be to grade extrinsic branches by the number of times they are 
executed and their behaviour. These extrinsic branches may be further classified into those present in
critical sections (often executing) and those used solely for setup or initialization. With such 
classification we may be able to focus on the most offending types of such branches and explore the different solutions
, both architectural and software, that could mitigate them.

Another class of extrinsic branches that we encountered were those that are employed to run a large serial
portion of a task between two heavily-parallel portion on the Kernel. Of the few cases that we saw, most of
them were present to make the entire task run on the GPU. However, the serial portion often had no relation with
the rest of the task, and could have exploited the ILP on a CPU much better than on a GPU. These branches,
extrinsic for a very different reason, cannot be as clearly differentiated. Because a relatively small 
serial code is faster as a part of the Kernel code: the time lost in memory transfer to or from the host (CPU)
and the device (GPU) to start executing the next Kernel, may negate any benefit of running the serial part on the CPU.
Hence, we believe more exploration could be done to understand the meaning of \textsl{extrinsic} and \textsl{intrinsic}
in the huge variety of general-purpose GPU code that we have today.
}
