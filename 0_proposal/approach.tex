%----------------------------------------------------------------------------------------
%	APPROACH	
%----------------------------------------------------------------------------------------

\par{
The program for an algorithm assumes an implementation for the device running the program (CPU, GPU, SMT etc.), which in turn determines the control flow present. However, some control flow is inherent to the algorithm and cannot be eliminated by any practical hardware implementation. The branches associated with them will be called \textbf{intrinsic} branches.
}

\par{
Some branches are the result of mapping an algorithm onto a specific hardware . These branches will be called \textbf{extrinsic} branches. For instance, in a graphical processing routine increasing brightness of a picture, loops must be included in CPU code which cycles through all the pixels. The branches due to such loops are extrinsic, present only because of mapping onto CPU. GPU code of general purpose programs may also have extrinsic branches, as described in Section \ref{sec:motivation}  
} 

\par{Our approachs is organized as \textbf{milestones} towards achieveing the objective of the project:}

\begin{itemize}

\item \textbf{Classifying Branches for each benchmark algorithm}. The CPU code of the algorithm is first used to identify intrinsic branches. It is relatively simple to identify intrinsic branches in CPU code as loops can be identified easily and are the most common extrinsic branches present.
\\
It is assumed that the intrinsic branches map similarly to the GPU code. That is, the intrinsic control portion of the code maps to similar control structures to both CPU and GPU. Using this assumption, the extrinsic branches of the algorithms are also identified by enumerating all other branches in the GPU code.

\item \textbf{Measuring Performance loss due to extrinsic branches in GPU code for each algorithm}. The GPU code is simulated in a simulator and the additional cycles due to the identified extrinsic branches is counted for each algorithm. These cycle losses are then compared with the total execution time. This will tell us if the losses due to extrinsic branches are significant and can be an important metric to judge suitability of current GPU architectures to run general programs.

\item \textbf{Proposing Hardware changes in GPU architectures to minimize the losses due to extrinsic branches}. If the losses found in the previous milestone are significant (for instance, more than 5\%) then any light hardware change that can partially mitigate such losses will be beneficial. We will explore different changes to the shader pipeline, taking hints from CPU and vector pipelines and the knowledge gained in class, to minimize the losses. Since this depends on the outcome of the second milestone, we cannot say much about our approach in this milestone apriori. 

\end{itemize}
