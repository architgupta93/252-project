%----------------------------------------------------------------------------------------
%	INTRODUCTION
%----------------------------------------------------------------------------------------

\par{Graphics Processing Units (GPUs) are increasingly used for scientific and other compute-intensive tasks. Owing to the availability of a large number of parallel resources (shader pipelines) originally built to execute graphics programs, they can achieve orders of magnitude higher performance on certain data-parallel applications than a conventional CPU. And unlike Vector processors they are cheaper, less power consuming and easily implemented on mobile platforms.
}

\par{
However, since the shader cores were not designed for general purpose computing, there is a performance loss due to a significant amount of control flow. Additionally, unlike vector processors, most GPUs do not have a tightly coupled scalar core. Hence, in order to map a general or scientic problem on a GPU architecture, the scalar part of the program can be processed in either of the two ways:}

\begin{enumerate}
	\item the scalar part of the program is executed on the CPU and the parallel part on the GPU in a General Purpose GPU (GPGPU) system, or 
	\item the scalar (serial) code is executed on the GPU as part of the execution kernel thereby under-utilizing the parallel hardware avaliable.
\end{enumerate} 

\par{
While the former method involves significant data movement (the CPU and GPU have seperate memory sub-systems), the latter approach causes the inclusion of a significant amount of control flow to the existing kernels.}
