%----------------------------------------------------------------------------------------
%	INTRODUCTION
%----------------------------------------------------------------------------------------

\par{Graphics Processing Units (GPUs) are increasingly used for scientific and other compute-intensive tasks. Owing to the availability of large number of parallel resources (shader pipelines), they can achieve orders of magnitude higher performance on certain data-parallel applications than a conventional CPU.
}

\par{
However, since the shader cores were not designed for general purpose computing, there is performance loss due to a significant amount of control flow. Additionally, unlike vector processors, most GPUs do not have tightly coupled scalar core(s). Hence, to map general problems on a GPU, the scalar portion of the program can be executed on: (1) the CPU, and the parallel part on the GPU (thereby limiting performance due to inter-system data movement); or (2) the GPU as part of the execution kernel thereby under-utilizing the parallel hardware avaliable. The latter approach causes the inclusion of significant amount of control flow in existing kernels. Our objective is to investigate their effects on overall performance.}
